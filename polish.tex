%!TEX program = xelatex

\documentclass[polish,12pt]{article}
\usepackage[main=polish]{babel}
\usepackage[autostyle]{csquotes}
\usepackage[type={CC}, modifier={by}, version={4.0}]{doclicense}
\usepackage{fontspec}
\usepackage{geometry}
\usepackage[colorlinks=true, urlcolor=blue, linkcolor=black]{hyperref}
\usepackage{tipa}
\usepackage{xurl}

\geometry{a4paper, margin=2.5cm}

\begin{document}
\setmainfont{Doulos SIL}
\title{%
    Upro\.s\.cony język polski\\
    \large Uproszczony język polski}
\author{Natalia Łotocka}
\maketitle
\tableofcontents
\newpage
\section{Zmiany}
\begin{itemize}
    \item Zastąpienie \enquote{sz}, \enquote{cz}, \enquote{rz}, \enquote{ch}, \enquote{ó} na odpowiednio \enquote{\.s}, \enquote{\.c}, \enquote{ż}, \enquote{h}, \enquote{u}
    \item Dodanie \enquote{'} jako oznaczenie niestandardowego akcentu w wyrazie
    \item Kiedy dochodzi do ubezdźwięcznienia głosek, są one zapisywane jako ich bezdźwięczne odpowiedniki
    \item Kiedy dochodzi do udźwięcznienia głosek, są one zapisywane jako ich dźwięczne odpowiedniki
    \item Wszystkie przedrostki są pisane łącznie z dodanym myślnikiem
    \item Słowa, w których dwu lub trójznaki są wymawiane oddzielnie, są pisane z dwukropkiem (np. d:żewo)
    \item Kiedy dwuznaki \enquote{ni}, \enquote{ci}, \enquote{si}, \enquote{zi} i trójznak \enquote{dzi} są wymawiane jako odpowiednio [\textipa{\textltailn}], [\textipa{\t{tC}}], [\textipa{C}], [\textipa{\textctz}], [\textipa{\t{d\textctz}}] to pisane są jako \enquote{ń}, \enquote{ć}, \enquote{ś}, \enquote{ź}, \enquote{dź}, a kiedy są wymawiane jako [\textipa{\textltailn i}], [\textipa{\t{tC}i}], [\textipa{Ci}], [\textipa{\textctz i}], [\textipa{\t{d\textctz}i}], to są zapisywane jako \enquote{ńi}, \enquote{ći}, \enquote{śi}, \enquote{źi}, \enquote{dźi}
\end{itemize}
\newpage
\section{Przykłady}
\subsection{Powszechna Deklaracja Praw Człowieka (Artykuł 1 i 2)\texorpdfstring{\\}{} Pof\.sehna Deklaracja Praf \.Cłowieka (Artykuł 1 i 2)}
\begin{quote}
    W\.syscy ludźe rodzą się wolńi i ruwńi pod względem sfej godnośći i sfyh praf. Są ońi obdażeńi rozumem i sumieńem i powinńi postępować wobec innyh w duhu braterstfa. \bigskip \\
    Każdy \.cłowiek pośada f\.systkie prawa i wolnośći zawarte w ńińej\.sej Deklaracji bes względu na jakiekolwiek rużńice rasy, koloru, płći, języka, wyznańa, pogląduf polity\.cnyh i innyh, narodowośći, pohodzeńa społe\.cnego, majątku, urodzeńa lub jakiegokolwiek innego stanu. \\
    Nie wolno ponadto \.cyńić żadnej rużńicy w zależnośći od sytuacji polity\.cnej, prawnej lub międzynarodowej kraju lub ob\.saru, do kturego dana osoba p\.synależy, bes względu na to, czy dany kraj lub ob\.sar jest ńepodległy, czy te\.s podlega systemowi powierńictfa, ńe-żądźi się samodźelńe lub jest w jakikolwiek sposup ograńi\.cony f sfej ńepodległośći.
\end{quote}
\subsection{Manifest Partii Komunistycznej (Wstęp)\texorpdfstring{\\}{} Mańifest Partii Komuńisty\.cnej (Fstęp)}
\begin{quote}
    Widmo krąży po Europie - widmo komuńizmu. F\.systkie potęgi starej Europy połą\.cyły się f śfiętej nagonce p\.sećif temu widmu: papie\.s i car, Metternich i Guizot, francuscy radykałowie i ńemieccy policjanći. \\
    Gdźe jest taka partia opozycyjna, ktura by ńe-była ok\.sy\.cana za komuńisty\.cną p\.ses sfyh p\.sećiwńikuf będącyh u władzy? Gdźe jest taka partia opozycyjna, ktura by s kolei ńe-odwzajemńiała się piętnującym zażutem komuńizmu zaruwno bardźej od śebie postępowym p\.sedstawićielom, opozycji, jak i sfoim reakcyjnym p\.sećiwńikom? \\
    Dwojaki wńosek wypływa s tego faktu. \\
    Komuńizm jest ju\.s p\.ses w\.systkie potęgi europejskie uznany za potęgę. \\
    \.Cas ju\.s najwy\.s\.sy, aby komuńiśći wyłożyli otfarće wobec całego śfiata sfuj punkt widzeńa, sfoje cele, sfoje dążeńa i bajce o widmie komuńizmu p\.sećifstawili mańifest samej partii. \\
    F tym celu zebrali się w Londyńe komuńiśći najrużńej\.syh narodowośći i nakreślili następujący Mańifest, ktury opublikowany zostaje w językah: angielskim, francuskim, ńemieckim, włoskim, flamandzkim i duńskim.
\end{quote}
\newpage
\subsection{Konstytucja Rzeczypospolitej Polskiej (frag. preambuły)\texorpdfstring{\\}{} Konstytucja Że\.cypospo'litej Polskiej (frag. preambuły)}
\begin{quote}
    \begin{center}
        F trosce o byt i p\.sy\.słość na\.sej Oj\.cyzny, \\
        odzyskaf\.sy w 1989 roku możliwość suwerennego i demokraty\.cnego stanowieńa o Jej losie, \\
        my, Narut Polski - f\.syscy obywatele Że\.cypospo'litej, \\
        zaruwno wieżący w Boga \\
        będącego źrudłem prawdy, sprawiedliwośći, dobra i piękna, \\
        jak i ńe-podźelający tej wiary, \\
        a te uńiwersalne wartośći wywodzący z innyh źrudeł, \\
        ruwńi f prawah i f powinnośćiah wobec dobra fspulnego - Polski, \newline
        [...]
    \end{center}
\end{quote}
\newpage
\section{Materiały źródłowe}
\begin{itemize}
    \item Zgromadzenie Ogólne Organizacji Narodów Zjednoczonych. 1948. Powszechna Deklaracja Praw Człowieka. Paryż: Organizacja Narodów Zjednoczonych.\newline (\url{https://www.unesco.pl/fileadmin/user_upload/pdf/Powszechna_Deklaracja_Praw_Czlowieka.pdf})
    \item Karol Marks, Fryderyk Engels. 1848. Manifest Partii Komunistycznej. Warszawa: Studenckie Koło Filozofii Marksistowskiej (Uniwersytet Warszawski).\newline (\url{https://www.ce.uw.edu.pl/wp-content/uploads/2018/10/4.-kapitalizm_marks_engels_manifest-komunistyczny.pdf})
    \item Konstytucja Rzeczypospolitej Polskiej z dnia 2 kwietnia 1997 r. Preambuła.\newline (\url{https://www.sejm.gov.pl/prawo/konst/polski/kon1.htm})
\end{itemize}
\section{Licencja}
\doclicenseThis
\end{document}
